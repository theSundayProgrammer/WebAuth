\hypertarget{webauth}{%
\section{WebAuth}\label{webauth}}

\begin{quote}
A small project to illustrate the use of Redis for authentication and
authorisation.
\end{quote}

\hypertarget{context}{%
\subsection{Context}\label{context}}

While developing a website to take student role calls on the phone, I
decided to use MySQL for the student database and a Redis for
authentication and authorisation. I am using Openresty (NGINX) as my web
server. Hence all the code will be in Lua. The
\href{https://github.com/theSundayProgrammer/WebAuth}{project} is still
its alpha stage but for now logging with uid/pwd : joe3/password at
\href{https://test.norwestcomputing.com.au/new_class}{Norwest Computing}
will get you access to the attendance register for a fictional class. In
the first stage only I am publishing only the user authentication part.
In the next stage I will publish the actual \#\# Basic Authentication
Nginx provides a
\href{https://docs.nginx.com/nginx/admin-guide/security-controls/configuring-http-basic-authentication/}{simple
authentication schema}. It consists of a simple text file with user
names and hashed passwords. In the following configuration any web
client trying to access the \emph{/api} area will be prompted for a
password unless the client has alreday logged in location /api \{
auth\_basic ``Administrator's Area''; auth\_basic\_user\_file
/etc/apache2/.htpasswd; \ldots{}\ldots{}\ldots{}. \}

Here \emph{/etc/apache2/.htpasswd} is the text file containing user
names and passwords. This type of authentication will suffice for a
website with a small number of users. Authorisation, of the
all-or-nothing kind can be implemented by having different password
files for different locations. So if a user has to be removed the user
name must be removed from all the password files

\hypertarget{advanced-authentication}{%
\subsection{Advanced Authentication}\label{advanced-authentication}}

Using a datbase to store the username and password has some advantages;
the main one being concurerency (a database can be read and updated
concuurently). The other advantage would be that it is possible to
implement authorisation based on roles or groups as well. However a full
Role Based Access Control (RBAC) has to be implemented by the database
server.

\hypertarget{javascript-web-tokens}{%
\subsubsection{Javascript Web Tokens}\label{javascript-web-tokens}}

JWT (Javascript Web Tokens) is a convention used to save the
authentication details in a cookie. The user name and other details such
as time to expire is encrypted and stored in a cookie. Since a HTTP
request is stateless the cookie contains all the details about the user.
Thus if there are multiple servers are used for load balancing any
server can decoded the JWT token if the secret key to encrytpt it is
known.

Hence if a location needs a user to be authenticated then the user is
redirected to a logon page. On successful logon the user is then
redirected to the requested page. The implementation of this
functionality is present in `check\_access.lua'

\hypertarget{why-redis}{%
\subsection{Why Redis}\label{why-redis}}

Redis is a NOSQL key-value database. While the type of the key can be
only a string the type of a value can in addition to being a primitive
type like number, string, blob or hyperloglog, can also be a set, a map
or a sequence. A map in Redis is called a HMAP or hasp-map, indicating
its implementation, and a sequence is called a list. For details refer
to Redis.

Notice that the three data structures: set, map and sequence cover
almost all dat structure requirements. Redis does not provide a
recursive data structure. A set cannot contain another set. With some
discipline though we could achieve the same result by having a set of
keys with each key refering to another set or map or sequence.

\hypertarget{authentication}{%
\subsection{Authentication}\label{authentication}}

User names and hashed passwords are store in a map whose key is
``users:passwords''. One of the first design issues that needs to be
addressed is nomenclature of keys. I use the convention that the
prefixi, ``users'' in this case is plural when it refers to the
collection `users' and singular if we need to create a key for a
specific user say, ``user:joe.''

The password may be hashed as follows:

\begin{Shaded}
\begin{Highlighting}[]
\KeywordTok{local} \KeywordTok{function}\NormalTok{ hash_pwd}\OperatorTok{(}\NormalTok{user}\OperatorTok{,}\NormalTok{pwd}\OperatorTok{)}  
\KeywordTok{local}\NormalTok{ sha256 }\OperatorTok{=} \FunctionTok{require}\StringTok{"resty.sha256"}
\CommentTok{-- create a private closure for calculating digest of single string}
  \KeywordTok{local}\NormalTok{ chunk }\OperatorTok{=}\NormalTok{ sha256}\OperatorTok{:}\NormalTok{new}\OperatorTok{()} 
  \KeywordTok{local}\NormalTok{ seed}\OperatorTok{=}\StringTok{"hsGtghLTh5fglo6d"} \CommentTok{-- secret prefix}
\NormalTok{  chunk}\OperatorTok{:}\NormalTok{update}\OperatorTok{(}\NormalTok{user}\OperatorTok{)}
\NormalTok{  chunk}\OperatorTok{:}\NormalTok{update}\OperatorTok{(}\NormalTok{seed}\OperatorTok{)} 
\NormalTok{  chunk}\OperatorTok{:}\NormalTok{update}\OperatorTok{(}\NormalTok{pwd}\OperatorTok{)}               
  \ControlFlowTok{return}\NormalTok{ str}\OperatorTok{.}\NormalTok{to_hex}\OperatorTok{(}\NormalTok{chunk}\OperatorTok{:}\NormalTok{final}\OperatorTok{())}
\ControlFlowTok{end}
\end{Highlighting}
\end{Shaded}

The following helper function wraps the openning and closing of
connection for a redis command

\begin{Shaded}
\begin{Highlighting}[]
\KeywordTok{local} \KeywordTok{function}\NormalTok{ exec}\OperatorTok{(}\NormalTok{func}\OperatorTok{)}
  \KeywordTok{local}\NormalTok{ red }\OperatorTok{=}\NormalTok{ redis}\OperatorTok{:}\NormalTok{new}\OperatorTok{()}

\NormalTok{  red}\OperatorTok{:}\NormalTok{set_timeouts}\OperatorTok{(}\DecValTok{1000}\OperatorTok{,} \DecValTok{1000}\OperatorTok{,} \DecValTok{1000}\OperatorTok{)} \CommentTok{-- 1 sec}
  \FunctionTok{assert}\OperatorTok{(}\NormalTok{ red}\OperatorTok{:}\FunctionTok{connect}\OperatorTok{(}\StringTok{"127.0.0.1"}\OperatorTok{,} \DecValTok{6379}\OperatorTok{))}
  \KeywordTok{local}\NormalTok{ results}\OperatorTok{,}\NormalTok{ err }\OperatorTok{=}\NormalTok{ func}\OperatorTok{(}\NormalTok{red}\OperatorTok{)}
\NormalTok{  red}\OperatorTok{:}\NormalTok{set_keepalive}\OperatorTok{(}\DecValTok{1000}\OperatorTok{,}\DecValTok{100}\OperatorTok{)}
    \ControlFlowTok{return}\NormalTok{ results}\OperatorTok{,}\NormalTok{err}
\ControlFlowTok{end}
\end{Highlighting}
\end{Shaded}

Hence a verify password function is more easily implemented as follows

\begin{Shaded}
\begin{Highlighting}[]
\KeywordTok{local}\NormalTok{ usertable}\OperatorTok{=}\StringTok{"users:passwords"}
\KeywordTok{function}\NormalTok{ verify_pwd}\OperatorTok{(}\NormalTok{user}\OperatorTok{,}\NormalTok{pwd}\OperatorTok{)}
  \KeywordTok{local}\NormalTok{ compute }\OperatorTok{=} \KeywordTok{function}\OperatorTok{(}\NormalTok{red}\OperatorTok{)}       
    \ControlFlowTok{return}\NormalTok{  red}\OperatorTok{:}\NormalTok{hmget}\OperatorTok{(}\NormalTok{usertable}\OperatorTok{,}\NormalTok{user}\OperatorTok{)}        
  \ControlFlowTok{end}     
  \KeywordTok{local}\NormalTok{ results}\OperatorTok{,}\NormalTok{err }\OperatorTok{=}\NormalTok{ exec}\OperatorTok{(}\NormalTok{compute}\OperatorTok{)}
  \ControlFlowTok{if}\NormalTok{  results }\ControlFlowTok{then}
    \ControlFlowTok{return}\NormalTok{ hash_pwd}\OperatorTok{(}\NormalTok{user}\OperatorTok{,}\NormalTok{pwd}\OperatorTok{)==}\NormalTok{results}\OperatorTok{[}\DecValTok{1}\OperatorTok{]} \KeywordTok{and} \DecValTok{1} \KeywordTok{or} \DecValTok{0}
  \ControlFlowTok{else}                    
    \ControlFlowTok{return}\NormalTok{ results}\OperatorTok{,}\NormalTok{err}
  \ControlFlowTok{end}                                  
\ControlFlowTok{end}
\end{Highlighting}
\end{Shaded}

A simple usage example is shown below:

\begin{Shaded}
\begin{Highlighting}[]
\KeywordTok{local}\NormalTok{ usrauth}\OperatorTok{=}\FunctionTok{require} \StringTok{"check_redis"}
\KeywordTok{local}\NormalTok{ user}\OperatorTok{=}\StringTok{"joe3"}
\KeywordTok{local}\NormalTok{ password}\OperatorTok{=}\StringTok{"password"}
\KeywordTok{local}\NormalTok{ verified}\OperatorTok{,}\NormalTok{err }\OperatorTok{=}\NormalTok{ usrauth}\OperatorTok{.}\NormalTok{verify_pwd}\OperatorTok{(}\NormalTok{user}\OperatorTok{,}\NormalTok{password}\OperatorTok{)}
\ControlFlowTok{if}\NormalTok{ verified }\KeywordTok{and}\NormalTok{ verified}\OperatorTok{==}\DecValTok{1} \ControlFlowTok{then}
\CommentTok{--success}
\ControlFlowTok{else}
\CommentTok{--fail}
\ControlFlowTok{end}
\end{Highlighting}
\end{Shaded}

The list of functions available are:

\begin{itemize}
\tightlist
\item
  \textbf{add\_role}\_(user\_name,role)\_ adds a role to the set
  ``user:''
\item
  \textbf{del\_user\_role}\_(user\_name,role)\_ deletes a role from the
  set ``user:''
\item
  \textbf{get\_assets}\_()\_ gets the set ``users:assets''
\item
  \textbf{get\_users}\_()\_ gets the set of keys of the map
  ``users:passwords''
\item
  \textbf{add\_asset}\_(asset\_name,role)\_ adds ``role'' to set
  ``assets:''
\item
  \textbf{del\_asset}\_(asset)\_ deletes the ``asset'' from the set
  ``users:assets'' and deletes the key ``asset:''
\item
  \textbf{del\_user}\_(user\_name)\_ deletes the key ``'' from the map
  ``users:passwords'' and deletes the key-value pair with key ``user:''
\item
  \textbf{add\_user}\_(user,pwd)\_ adds key-value
  \textless{}user,hashed\_pwd\textgreater{} to the map
  ``users:passwords'' Notice that if the user exists already its
  password is overwritten
\item
  \textbf{verify\_pwd}\_(user,pwd)\_ verify the password as shown above
\item
  \textbf{user\_auth}\_(user\_name,asset\_name)\_ Computes the
  intersection of ``user:'' and ``asset:'' and returns true if the set
  contains at least one element
\item
  \textbf{del\_roles}\_(role)\_ ``role'' is removed from the set
  ``users:roles'', and from every set ``user:*'' and ``asset:*''
\item
  \textbf{get\_roles}\_()\_ gets the list of all roles
\end{itemize}
